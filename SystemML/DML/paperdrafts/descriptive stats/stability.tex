
{\em Numerical stability} refers to the inaccuracies in computation resulting from finite precision floating point arithmetic on digital computers with round-off and truncation errors. Multiple algebraically equivalent formulations of the same numerical calculation often produce very different results. While some methods magnify these errors, others are more robust or stable. The exact nature and the magnitude of these errors depend on several different factors like the number of bits used to represent floating point numbers in the underlying architecture (commonly known as {\em precision}), the type of computation that is being performed, and also on the number of times a particular operation is performed. Such round-off and truncation errors typically grow with the input data size. Given the exponential growth in the amount of data that is being collected and processed in recent years, numerical stability becomes an important issue for many practical applications. 

One possible strategy to alleviate these errors is to use special software packages that can represent floating point values with {\em arbitrary} precision. For example, $BigDecimal$ in Java provides the capability to represent and operate on arbitrary-precision signed decimal numbers. Each $BigDecimal$ number is a Java object. While the operations like addition and subtraction on native data types (double, int) are performed on hardware, the operations on $BigDecimal$ are implemented in software. Furthermore, JVM has to explicitly manage the memory occupied by $BigDecimal$ objects. Therefore, these software packages often suffer from significant performance overhead. In our benchmark studies, we observed up to 2 orders of magnitude slowdown for addition, subtraction, and multiplication using $BigDecimal$ with precision $1000$ compared to the native $double$ data type; and up to 5 orders of magnitude slowdown for division.

In the rest of this section, we discuss methods adopted in SystemML to calculate descriptive statistics, which are both numerically stable and computationally efficient. We will also highlight the common pitfalls that must be avoided in practice. First, we discuss the fundamental operations {\em summation} and {\em mean}, and subsequently present the methods that we use for computing {\em higher-order statistics} and {\em covariance}.

%In the context of large distributed MapReduce-style environments, one needs effective algorithms that can both exploit massive data parallelism and produce robust results with the increasing number of input data items. 


%Most of descriptive statistics, except for order statistics, can be expressed in certain summation form, thus may appear simple to be implemented in MapReduce. However, straightforward implementations can sometimes lead to disasters in numerical accuracy, due to round-off and truncation errors associated with finite precision arithmetic. As the need for analyzing large volumes of data increases, \textit{numerical stability} becomes even more critical.

%In the MapReduce environment, one needs effective algorithms that can both exploit massive data parallelism and produce robust results with growing data sizes. Unfortunately, all the existing large scale platforms, such as  PIG~\cite{pig}, HIVE~\cite{hive} and JAQL~\cite{jaql}, are still using numerical unstable algorithms to compute basic statistics. For example, PIG as of version 0.9.1 only provides the very basic \textit{sum} and \textit{mean} functions, and neither of them are computed using numerically stable algorithms. The recent version of HIVE (0.7.1) supports more statistics including \textit{sum}, \textit{mean}, \textit{variance}, \textit{standard deviation}, \textit{covariance}, and \textit{corelation}. Luckily, the developers who implemented \textit{variance}, \textit{standard deviation}, \textit{covariance}, and \textit{corelation} are using stable algorithms similar to our approaches. However, \textit{sum} and \textit{mean} are still computed using numerically unstable algorithms. Some of these large scale systems support BigDecimal type, but it incurs heavy performance overhead, thus is not recommended for large scale data processing. In our experiments, we observed that simple addition, subtraction and multiplication using BigDecimal is 2 orders of magnitude slower than the equivalent operations using double. And division in BigDecimal can be 5 orders of magnitude slower than double with a precision 1000.  
%Although some of these systems support BigDecimal types, using BigDecimal for numerical computation brings significant performance overhead, thus is not recommended for large scale data processing.}. 
%Numerically unstable algorithms are still in use, even in well-known distributed database products~\cite{teradata}. 

%In this section, we use summation and central moment as the driving examples to highlight the common pitfalls in implementing descriptive statistics, and describe how numerical stability is achieved in SystemML.

%
%Numerical instability refers to inaccuracies in computation resulting from finite precision floating point arithmetic on digital computers with round-off and truncation errors. Quite often, multiple algebraically equivalent formulations of the same numerical calculation produce very different results -- some may propagate and magnify these approximation errors whereas others may be more robust or numerically stable. As the need for analyzing large volumes of data increases, the issue of numerical stability becomes much more critical.
%In the context of large distributed MapReduce-style environments, one needs effective algorithms that can both exploit massive data parallelism and produce robust results with the increasing number of input data items. 

%To the best of our knowledge, none of the existing large scale data processing platforms such as PIG~\cite{pig}, HIVE~\cite{hive} and JAQL~\cite{jaql} pay attention to these issues\footnote{Although some of these systems support BigDecimal types, using BigDecimal for numerical computation brings significant performance overhead, thus is not recommended for large scale data processing.}. In this section, we use summation and arbitrary order central moment as the driving examples, to highlight the common pitfalls of implementing descriptive statistics and describe how numerical stability is achieved in SystemML.

%Numerical instability is a problem caused by the finite precision arithmetic on numerical algorithms. If there were no round-off or truncation errors associated with finite precision arithmetic, numerical algorithms would always approach the right solution. Quite often, different algebraically equivalent ways of implementing a numerical calculation will result in very different result, with some propagating and magnifying approximation errors and others more robust -- also called numerically stable. The goal of most numerical analysis is to select these numerically stable algorithms. 

\subsection{Stable Summation}
\label{sec:sum}

Summation is a fundamental operation in many statistical functions, such as mean, variance, and norms. The simplest method is to perform {\em naive recursive summation}, which initializes $sum=0$ and incrementally updates $sum$. It however suffers from numerical inaccuracies even on a single computer. For instance, with $2$-digit precision, naive summation of numbers $1.0, 0.04, 0.04, 0.04, 0.04, 0.04$ results in $1.0$, whereas the exact answer is $1.2$. This is because once the first element $1.0$ is added to $sum$, adding $0.04$ will have no effect on $sum$ due to round-off error. %Both PIG and HIVE use this naive recursive algorithm to compute sum. 
A simple alternative strategy is to first sort the data in increasing order, and subsequently perform the naive recursive summation. While it produces the accurate result for the above example, it is only applicable for non-negative numbers, and more importantly, it requires an expensive sort. 

%There exist a number of other algorithms for stable summation~\cite{numStabBook}. One notable technique is proposed by Kahan~\cite{kahan1965further}. It is the recursive summation with a correction term to reduce the rounding errors. Algorithm~\ref{algo:kahan} shows the incremental update rule for Kahan algorithm. 

There exists a number of other methods for stable summation~\cite{numStabBook}. One notable technique is proposed by Kahan~\cite{kahan1965further}. It is a {\em compensated summation} technique -- see Algorithm~\ref{algo:kahan}. This method maintains a {\em correction} or {\em compensation} term to accumulate errors encountered in naive recursive summation. 

\SmallCrunch
\begin{algorithm}
\caption{Kahan Summation Incremental Update} 
\label{algo:kahan}
\begin{algorithmic}
\small{
\STATE //  $s_1$ and $s_2$ are partial sums, $c_1$ and $c_2$ are correction terms
\STATE \textsc{KahanIncrement}($s_1, c_1, s_2, c_2$)\{
\STATE \ \ $corrected\_s_2 = s_2 + (c_1+c_2)$
\STATE \ \ $sum=s_1+corrected\_s_2$
\STATE \ \ $correction=corrected\_s_2-(sum - s_1)$
\STATE \ \ \textbf{return} $(sum, correction)$ \}
}
\end{algorithmic}
\end{algorithm}
\Crunch

%\begin{algorithm}
%\caption{Kahan Summation} 
%\label{algo:kahan}
%\begin{algorithmic}
%\small{
%\STATE \textbf{KahanSum}(X: input data items)\{
%\STATE \ \ $sum=0$, $correction=0$
%\STATE \ \ \textbf{for} $i = 1 \to X.size$ 
%\STATE \ \ \ \ $(sum, correction)=\textbf{KahanAdd}(sum, correction, x_i, 0)$
%\STATE \ \ \textbf{return} $sum$ \}
%\STATE 
%\STATE //  $s_1$ and $s_2$ are partial sums, $c_1$ and $c_2$ are correction terms
%\STATE \textbf{KahanAdd}($s_1, c_1, s_2, c_2$)\{
%\STATE \ \ $corrected\_s_2 = s_2 + c_1+c_2$
%\STATE \ \ $sum=s_1+corrected\_s_2$
%\STATE \ \ $new\_c=corrected\_s_2-(sum - s_1)$
%\STATE \ \ \textbf{return} $(sum, new\_c)$ \}
%}
%\end{algorithmic}
%\end{algorithm}


Kahan and Knuth independently proved that Algorithm~\ref{algo:kahan} has the following relative error bound~\cite{numStabBook}: 
\begin{equation}
\begin{small}
\label{eq:relerror}
\frac{|E_n|}{|S_n|}=\frac{|\hat{S}_n-S_n|}{|S_n|}\leq (2u+O(nu^2))\condnumber,
\end{small}
\end{equation}
where $S_n=\sum\limits_{i=1}^n x_i$ denotes the true sum of a set of $n$ numbers $X=\{x_1, x_2,...,x_n\}$, and $\hat{S}_n$ is the sum produced by the summation algorithm, $u=\frac{1}{2}\beta^{1-t}$ is the {\em unit roundoff} for a floating point system with base $\beta$ and precision $t$. It denotes the upper bound on the relative error due to rounding. For IEEE 754 floating point standard with $\beta=2$ and $t=53$, $u = 2^{-53}\approx 10^{-16}$. Finally, $\condnumber$ is known as the {\em condition number} for the summation problem, and it is defined as the fraction $\frac{\sum\limits_{i=1}^n|x_i|}{|\sum\limits_{i=1}^n x_i|}$. The condition number measures the sensitivity of the problem to approximation errors, independent of the exact algorithm used. Higher the value of $\condnumber$, the higher will be the numerical inaccuracies and the relative error. It can be shown that for a random set of numbers with a nonzero mean, the condition number of summation asymptotically approaches to a finite constant as $n\to\infty$. Evidently, the condition number is equal to $1$ when all the input values are non-negative. It can be seen from Equation~\ref{eq:relerror} that when $nu\leq 1$, the bound on the relative error is independent of input size $n$. In the context of IEEE 754 standard, it means that when $n$ is in the order of $10^{16}$ the relative error can be bounded independent of the problem size. In comparison, the naive recursive summation has a relative error bound $\frac{|E_n|}{|S_n|}\leq (n-1)u\condnumber +\frac{O(u^2)}{|\sum\limits_{i=1}^n x_i|}$~\cite{numStabBook}, which clearly is a much larger upper bound when compared to Equation~\ref{eq:relerror}.

%It is proved in~\cite{numStabBook} that this algorithm has a relative error bound $\frac{|E_n|}{|S_n|}=\frac{|\hat{S}_n-S_n|}{|S_n|}\leq (2u+O(nu^2))\condnumber$, where $S_n=\sum\limits_{i=1}^n x_i$ is the true sum of a set of $n$ numerical values $X=\{x_1, x_2,...,x_n\}$, $\hat{S}_n$ is the result produced by the summation algorithm, $\condnumber =\frac{\sum\limits_{i=1}^n|x_i|}{|\sum\limits_{i=1}^n x_i|}$, and $u$ is the \textit{unit roundoff}. $u=\frac{1}{2}\beta^{1-t}$ for a floating point system with base $\beta$ and precision $t$. For IEEE double with $\beta=2$ and $t=53$, $u=2^{-53}\approx 10^{-16}$. $\condnumber $ is often called the \textit{condition number} of a summation problem. It measures the sensitivity of the sum for a given data set, independent of the algorithm used. It is proved that for random inputs with a nonzero mean the condition number asymptotically approaches to a finite constant as $n\to\infty$. Obviously, when inputs are all non-negative, the condition number is 1. Therefore, given a constant condition number, when $nu\leq 1$, the relative error is in the precision of $u$, which is independent of the number of data items $n$. For IEEE double, this means when $n\leq 10^{16}$, the relative error is independent of problem size. In comparison, the naive recursive summation has a relative error bound $\frac{|E_n|}{|S_n|}\leq (n-1)u\condnumber +O(u^2)$, which increases linearly with the problem size~\cite{numStabBook}.


%\textbf{Stable Summation.} Summation is a fundamental operation in many statistical functions, such as mean, variance, and norms. When performed using the simplest method, it suffers from numerical inaccuracies even on a single computer. Consider the summation of the following six numbers: $1.0, 0.04, 0.04, 0.04, 0.04, 0.04$. With 2 digit precision, the exact answer should be $sum=1.2$. However, the \textit{naive recursive summation} algorithm, which initializes $sum=0$ and keeps $sum=sum+x_i$, will results in $sum=1.0$. This is because once the first element $1.0$ is added to $sum$, adding $0.04$ will have no effect on $sum$ due to round-off error. %This naive algorithm is used in Pig, Hive and JAQL.

%Summations of numerical values are ubiquitous in statistical analysis. They appear in all kinds of statistical functions, such as mean, variance, norms and so on. As illustrated in the example below, even calculating the accurate summation of a set of numbers in a single machine is very tricky. Consider the summation of the following six numbers: $1.0, 0.04, 0.04, 0.04, 0.04, 0.04$. With 2 digit precision, the exact answer should be $sum=1.2$. However, the \textit{naive recursive summation} algorithm, which initializes $sum=0$ and keeps $sum=sum+x_i$, will results in $sum=1.0$. This is because once the first element $1.0$ is added to the sum, adding $0.04$ will have no effect on the sum due to round-off error.

%The core of the Kahan summation is an incremental update algorithm, shown in Algorithm~\ref{algo:kahan}, which aggregates two partial sums with their associated correction terms\footnote{For a single data item, the sum is its value and the correction term is 0.}. 

%This algorithm, when run on a single machine, has an error bound $|E_n|=|\hat{S}_n-S_n|\leq (2u+O(nu^2))\sum\limits_{i=1}^n|x_i|$, where $S_n$ is the true sum of $n$ numerical values $x_1, x_2,...,x_n$, $\hat{S}_n$ is the result produced by the summation algorithm, and $u$ is the \textit{unit roundoff}.  $u=\frac{1}{2}\beta^{1-t}$ for a floating point system with base $\beta$ and precision $t$. For IEEE double with $\beta=2$ and $t=53$, $u=2^{-53}\approx 10^{-16}$. When $nu\leq 1$, the error is independent of the number of data items.

%To produce accurate summation for large volumes of data on MapReduce, we need an algorithm that is easy to be parallelized and robust with increasing number of input data items. There have been well-studied summation algorithms in numerical analysis~\cite{numStabBook}. For non-negative numbers, recursively adding up the numbers by their ascending order will result in more accurate sum than the naive recursive summation. For example, this approach will result in the right summation $1.2$ for the above sequence of numbers. But this \textit{ordered recursive sum} requires first an expensive full sort on the set of data and then another scan of the data. Luckily, there is a numerically stable summation algorithm, called \textit{Kahan summation} that just requires one scan of the data. The \textit{Kahan summation} algorithm, shown in Algorithm~\ref{algo:kahan}, is a compensated summation method, which is recursive summation with a correction term to reduce the rounding errors. This algorithm, when run on a single machine, has an error bound $|E_n|=|\hat{S}_n-S_n|\leq (2u+O(nu^2))\sum\limits_{i=1}^n|x_i|$, where $S_n$ is the true sum of $n$ numerical values $x_1, x_2,...,x_n$, $\hat{S}_n$ is the result produced by the summation algorithm, and $u$ is the \textit{unit roundoff}.  $u=\frac{1}{2}\beta^{1-t}$ for a floating point system with base $\beta$ and precision $t$. For IEEE double with $\beta=2$ and $t=53$, $u=2^{-53}\approx 10^{-16}$. When $nu\leq 1$, the error is independent of the number of data items.

One can easily extend the Kahan algorithm to the MapReduce setting. 
%The Kahan summation algorithm can be easily adapted to the MapReduce environment. 
The resulting algorithm % shown in Algorithm~\ref{algo:mrkahan}, 
is a MapReduce job in which each mapper applies \textsc{KahanIncrement} and generates a partial sum with correction, and a single reducer produces the final sum. % by applying \textsc{KahanIncrement}. 
%Through error analysis, we can derive that when each mapper processes $\leq 10^{16}$ data items, this algorithm is robust w.r.t. the total number of data items to be summed using IEEE doubles (proof omitted). 

Through error analysis, we can derive the relative error bound for this MapReduce Kahan summation algorithm: $\frac{|E_n|}{|S_n|}\leq [4u+4u^2+O(mu^2)+O(\frac{n}{m}u^2)+O(mu^3)+O(\frac{n}{m}u^3)+O(nu^4)]\condnumber $ (see Appendix for proof). Here, $m$ is the number of mappers ($m$ is at most in 1000s). As long as $\frac{n}{m}u\leq 1$ (and $m \ll n$), the relative error is independent of the number of input data items. In the context of IEEE 754 standard, it can be shown that when $\frac{n}{m}$ is in the order of $2^{53} \approx 10^{16}$, the relative error can be bounded independent of $n$. In other words, as long as the number of elements processed by each mapper is in the order of $10^{16}$, the overall summation is robust with respect to the total number of data items to be summed. Therefore, by partitioning the work across multiple mappers, the MapReduce summation method is able to scale to larger data sets while keeping the upper bound on relative error independent of the input data size $n$.

%Following the error analysis for Kahan summation, we can easily derive the error bound for this MapReduce Kahan summation algorithm: $E_n\leq [4u+4u^2+O(mu^2)+O(\frac{n}{m}u^2)+O(\frac{n}{m}u^3)+O(mu^3)+O(nu^4)]\sum\limits_{i=1}^n|x_i|$. (Proof is omitted in the interest of space.) Here, $m$ is the number of mappers used in the MapReduce job ($m < n$). As long as $\frac{n}{m}u\leq 1$, the error is independent of number of input data items. For IEEE double, this means $\frac{n}{m}\leq 2^{53} \approx 10^{16}$. In other words, when each mapper process $\leq 10^{16}$ data items, the algorithm is robust with respect to the total number of data items to be summed.

%As a simple example, consider the sum of a sequence of numbers $1.0, 0.04, 0.04, 0.04, 0.04, 0.04$. With 2 digits precision, the correct answer should be $sum=1.2$. However, the naive \textit{recursive summation} algorithm, which initializes $sum=0$ and keeps $sum=sum+x_i$, will results in $sum=1.0$. This is because once the first element $1.0$ is added to the sum, adding $0.04$ will have no effect on the sum, due to round-off error. But if we reorder the sequence in ascending order $0.04, 0.04, 0.04, 0.04, 0.04, 1.0$, then the recursive summation algorithm will generate the correct result $sum=1.2$. 


%Table~\ref{tab:sum} lists these algorithms and their error bound analysis in the sequential environment. The \textit{recursive sum} is the naive algorithm described before. The \textit{pairwise sum} algorithm recursively performs pair-wise addition, reducing the array size by a factor of two in each recursion. The \textit{Kahan sum} algorithm is a compensated summation method, which is recursive summation with a correction term to reduce the rounding errors. It is described in Algorithm~\ref{algo:kahan}. In the error analysis shown in Table~\ref{tab:sum}, $S_n$ is the true sum of $n$ numerical values $x_1, x_2,...,x_n$, $\hat{S}_n$ is the result produced by a summation algorithm, $E_n=\hat{S}_n-S_n$ is the error of a summation algorithm, $u$ is the \textit{unit roundoff}, $u=\frac{1}{2}\beta^{1-t}$ for a floating point system with base $\beta$ and precision $t$. For IEEE double with $\beta=2$ and $t=53$, $u=2^{-53}\approx 10^{-16}$.

%The MapReduce algorithm is shown in Algorithm~\ref{algo:mrkahan}. Error bound is 
%$E_n\leq [4u+4u^2+O(mu^2)+O(\frac{n}{m}u^2)+O(\frac{n}{m}u^3)+O(mu^3)+O(nu^4)]\sum\limits_{i=1}^n|x_i|$. Here, $m$ is the number of mappers used in the MapReduce job ($m < n$). As long as $\frac{n}{m}u\leq 1$, the error is independent of input data size. For IEEE double, this means $\frac{n}{m}\leq 2^{53} \approx 10^{16}$.

%\begin{algorithm}
\caption{MapReduce Kahan Summation} 
\label{algo:mrkahan}
\begin{algorithmic}
\small{
\STATE //helper function
\STATE //$s_1$ and $s_2$ are partial sums, $c_1$ and $c_2$ are correction terms
\STATE \textbf{KahanAdd}($s_1, c_1, s_2, c_2$)\{
\STATE \ \ $corrected\_s_2 = s_2 + c_1+c_2$
\STATE \ \ $sum=s_1+corrected\_s_2$
\STATE \ \ $new\_c=corrected\_s_2-(sum - s_1)$
\STATE \ \ \textbf{return} $(sum, new\_c)$ \}
\STATE 
\STATE // Mapper Class
\STATE \textbf{configure}()\{ $sum=0$, $correction=0$ //initialization \}
\STATE \textbf{map}(k: the dummy key, v: the value)\{
\STATE \ \ $(sum, correction)=\textbf{KahanAdd}(sum, correction, v, 0)$ \}
\STATE \textbf{close}()\{ \textbf{emit} $<null, (sum, correction)>$ \}
\STATE 
\STATE // Reducer Class
\STATE \textbf{reduce}(k: the dummy key, listV: the list of values)\{
\STATE \ \ $sum=0$, $correction=0$
\STATE \ \ \textbf{for} $(s, c)$ in  listV 
\STATE \ \ \ \ $(sum, correction)=\textbf{KahanAdd}(sum, correction, s, c)$
\STATE \ \ \textbf{emit} $<null, sum>$ \}
}
\end{algorithmic}
\end{algorithm}



%\begin{table}[t]
%\centering
%\begin{tabular}{|c|l|}
%\hline
%Algorithm & Error Bound: $|E_n|=|\hat{S}_n-S_n|$\\
%\hline
%Recursive Sum & $\leq (n-1)u\sum\limits_{i=1}^n|x_i|+O(u^2)$\\
%\hline
%Pairwise Sum & $\leq {\frac{u\log{n}}{1-u\log{n}}}\sum\limits_{i=1}^n|x_i|$ \\
%\hline
%Kahan Sum & $\leq (2u+O(nu^2))\sum\limits_{i=1}^n|x_i|$ \\
%\hline
%\end{tabular}
%\caption{Sequential summation algorithms and their error analysis. }
%\label{tab:sum}
%\end{table}

\subsection{Stable Mean}
\label{sec:mean}

Mean is a fundamental operation for any quantitative data analysis. The widely used technique is to divde the sum by the total number of input elements. %Both PIG and HIVE relies on this approach, where $sum$ is computed using the naive recursive summation algorithm. 
This straightforward method of computing mean however suffers from numerical instability. As the number of data points increases, the accuracy of sum decreases, thereby affecting the quality of mean. Even when the stable summation is used, sum divided by count technique often results in less accurate results. 

In SystemML, we employ an incremental approach to compute the mean. This method maintains a running mean of the elements processed so far. It makes use of the update rule in Equation~\ref{eq:mean}. In a MapReduce setting, all mappers apply this update rule to compute the partial values for count and mean. These partial values are then aggregated in the single reducer to produce the final value of mean.

\begin{small}
\begin{equation}
\begin{split}
n=& n_a+n_b, \;\; \delta=\mu_b-\mu_a, \;\; \mu=\mu_a\varoplus n_b\frac{\delta}{n}
\end{split}
\label{eq:mean}
\end{equation}
\end{small}

In Equation~\ref{eq:mean}, $n_a$ and $n_b$ denote the partial counts, $\mu_a$ and $\mu_b$ refer to partial means. The combined mean is denoted by $\mu$, and it is computed using the \textsc{KahanIncrement} function in Algorithm~\ref{algo:kahan}, denoted as $\varoplus$ in the equation. In other words, we keep a correction term for the running mean, and $\mu=\mu_a\varoplus n_b\frac{\delta}{n}$ is calculated via $(\mu.value, \mu.correction)=$ \textsc{KahanIncrement} $(\mu_a.value, \mu_a.correction, n_b\frac{\delta}{n}, 0)$. Note that the use of \textsc{KahanIncrement} is important to obtain stable value for $\mu$. When the value of $n_b\frac{\delta}{n}$ is much smaller than $\mu_a$, the resulting $\mu$ will incur a loss in accuracy -- which can be alleviated by using \textsc{KahanIncrement}. As we show later in Section~\ref{sec:exp}, this incremental algorithm results in more accurate results. We also use it in stable algorithms to compute higher order statistics and covariance (see Sections~\ref{sec:highorder}~\&~~\ref{sec:covariance}).



\subsection{Stable Higher-Order Statistics}
\label{sec:highorder}

We now describe our stable algorithms to compute higher-order statistics, such as variance, skewness and kurtosis. The core computation is to calculate the $p^{th}$ central moment $m_p=\frac{1}{n}\sum\limits_{i=1}^{n}(x_i-\bar{x})^p$. Central moment can be used to describe higher-order statistics. For instance, variance $\sigma^2=\frac{n}{n-1}m_2$, skewness $g_1=\frac{m_3}{m_2^{1.5}}\cdot(\frac{n-1}{n})^{1.5}$, and kurtosis $g_2=\frac{m_4}{m_2^{2}}\cdot(\frac{n-1}{n})^{2}-3$. The standard two-pass algorithm produces relatively stable results (for $m_2$, the relative error bound is $nu+n^2u^2\condnumber ^2$, where $\condnumber = (\frac{\sum\limits_{i=1}^{n}x_i^2}{\sum\limits_{i=1}^{n}(x_i-\bar{x})^2})^{\frac{1}{2}}$ is the condition number), but it requires two scans of data -- one scan to compute $\bar{x}$ and the second to compute $m_p$. A common technique (pitfall) is to apply a simple textbook rewrite to get a one-pass algorithm. For instance, $m_2$ can be rewritten as $\frac{1}{n}\sum\limits_{i=1}^{n}x_i^2-\frac{1}{n^2}(\sum\limits_{i=1}^{n}x_i)^2$. The sum of squares and sum can be computed in a single pass. However, this algebraical rewrite is known to suffer from serious stability issues resulting from cancellation errors when performing subtraction of two large and nearly equal numbers (relative error bound is $nu\condnumber ^2$). This rewrite, as we show later in Section~\ref{sec:univariate_exp}, may actually produce a negative result for $m_2$, thereby resulting in grossly erroneous values for variance and other higher-order statistics. 
%Surprisingly, this notoriously unstable algorithm is still used in a popular distributed database platform~\cite{teradata} -- emphasizing the need for immediate attention to numerical stability issues in large scale data processing.

%A pitfall that people often run into is to apply a simple textbook rewrite rule to get a one-pass algorithm. For example, $2^{nd}$ central moment is rewritten as $m_2=\sum\limits_{i=1}^{n}x_i^2-\frac{1}{n}(\sum\limits_{i=1}^{n}x_i)^2$, where the sum as well as sum of squares are computed in a single pass. Even though the rewrite is algebraically equivalent, it is known to suffer from serious stability issues resulting from cancellation errors while performing the substraction of two large nearly equal numbers.

%Besides simple summation, descriptive statistics contain a number of high order statistics, such as central moment, variance, skewness and kurtosis. The $k^{th}$ central moment is defined as $m_k=\frac{1}{n}\sum\limits_{i=1}^{n}(x_i-\bar{x})^k$. Central moment is also the building block for the other high order statistics. For example, variance can be written as $s^2=\frac{n}{n-1}m_2$ and skewness can be expressed as $g_1=\frac{m_3}{m_2^{1.5}}$. Therefore, to produce efficient and numerically stable high order statistics, we need an efficient and numerically stable central moment algorithm. The standard two-pass algorithm using stable summation produces the numerically stable central moment, but requires two scans of data: the first to produce the mean and the second to calculate the central moment. A pitfall people often run into is to apply a simple textbook rewrite on the definition of central moment to get a one-pass algorithm. For example, $2^{nd}$ central moment can be rewritten as $m_2=\sum\limits_{i=1}^{n}x_i^2-\frac{1}{n}(\sum\limits_{i=1}^{n}x_i)^2$. Then this textbook one-pass algorithm can calculate the sum and sum of squares in one scan of the data followed by simply arithmetic on the results. As elaborated in~\cite{numStabBook}, this seeming mathematically equivalent and more efficient algorithm can lead to numerical disasters, due to the severe cancellation problem when performing a substraction on two large and nearly equal numbers. The computed $2^{nd}$ central moment (or variance) is often zero and sometimes even negative. But surprisingly, this notoriously unstable algorithm is still in use in some database products~\cite{netezza, teradata}.

In SystemML, we use a stable MapReduce algorithm that is based on an existing technique~\cite{cm}, to incrementally compute central moments of arbitrary order. It makes use of an update rule (Equation~\ref{eq:cmeq}) that combines partial results obtained from two disjoint subsets of data (denoted as subscripts $a$ and $b$). Here, $n$, $\mu$, $M_p$ refer to the cardinality, mean, and $n\times m_p$ of a subset, respectively. Again, $\varoplus$ represents addition through \textsc{KahanIncrement}. When $p=2$, the algorithm has a relative error bound of $nu\condnumber$. While there is no formal proof to bound the relative error when $p > 2$, we empirically observed that this method produces reasonably stable results for skewness and kurtosis. In the MapReduce setting, the same update rule is used in each mapper to maintain running values for these variables, and to combine partial results in the reducer. Note that the update rule in~\cite{cm} uses basic addition instead of the more stable \textsc{KahanIncrement}. In Section~\ref{sec:comparison_exp}, we will evaluate the effect of \textsc{KahanIncrement} on the accuracy of higher-order statistics.
 
%Note that HIVE utilizes a similar update rule to compute variance and standard deviation. However, HIVE uses the basic addition for the updates instead of the more stable \textsc{KahanIncrement}. In Section~\ref{sec:comparison_exp}, we will evaluate the effect of \textsc{KahanIncrement} by comparing the algorithms used in SystemML and HIVE for variance and standard deviation.

%\begin{equation}
%M_p=M_{p,a}+M_{p,b}+\sum\limits_{j=1}^{p-2}{p \choose j}[(-\frac{n_2}{n})^j M_{p-j,a}+(\frac{n_1}{n})^j M_{p-j,b}]\delta + (\frac{n_1n_2}{n}\delta)^p[\frac{1}{n_2^{p-1}}-(\frac{-1}{n_1})^{p-1}]$
%\STATE \ \ \textbf{return} $(n,\mu,M_2,..., M_k)
%\label{eq:cmeq}
%\end{equation}
\Crunch
\begin{small}
\begin{equation}
\begin{split}
n=& n_a+n_b, \;\; \delta=\mu_b-\mu_a, \;\; \mu=\mu_a\varoplus n_b\frac{\delta}{n} \\
M_p =& M_{p,a}\varoplus M_{p,b}\varoplus \{\sum\limits_{j=1}^{p-2}{p \choose j} [(-\frac{n_b}{n})^j M_{p-j,a} \\
  & + (\frac{n_a}{n})^j M_{p-j,b}]\delta^j + (\frac{n_an_b}{n}\delta)^p[\frac{1}{n_b^{p-1}}-(\frac{-1}{n_a})^{p-1}]\}
\end{split}
\label{eq:cmeq}
\end{equation}
\end{small}
\Crunch

%We adopt the algorithm introduced in~\cite{cm} to numerically stably calculate arbitrary order central moment. As shown in Algorithm~\ref{algo:cm}, the core of the algorithm is to calculate the value of $M_k=\sum\limits_{i=1}^{n}(x_i-\bar{x})^k$. Similar to the Kahan Sum algorithm, this algorithm can also be easily adapted to the MapReduce environment, by letting mappers compute the partial values and a single reducer produce the final result.

%\begin{algorithm}
\caption{Central Moment} 
\label{algo:cm}
\begin{algorithmic}
\small{
\STATE \textbf{CentralMoment}(X: input data items, k: order)\{
\STATE \ \ $n=0$, $\mu=0$, $M_2=0$..., $M_k=0$
\STATE \ \ \textbf{for} $i = 1 \to X.size$ 
\STATE \ \ \ \ $(n, \mu, M_2,...,M_k)=\textbf{incM}(n, \mu, M_2,...,M_k,1, x_i, 0,...,0)$
%\STATE \ \ \ \ \ \ \ \ \ \ \ \ \ \ \ \ \ \ \ \ \ \ \ \ \ \ \ \ \ \ \ \ \ \ \ \ \ \ \ \ \ \ \  $1, x_i, 0,...,0)$
\STATE \ \ \textbf{return} $\frac{M^k}{n}$ \}
\STATE 
\STATE \textbf{incM}($n_a,\mu_a,M_{2,a},..., M_{k,a}$, // $1^{st}$ input
\STATE \ \ \ \ \ \ $n_b,\mu_b,M_{2,b},..., M_{k,b}$)\{ // $2^{nd}$ input
\STATE \ \ $\textbf{if }n_1=0\textbf{ then return } (n_b,\mu_b,M_{2,b},..., M_{k,b})$
\STATE \ \ $\textbf{if }n_2=0\textbf{ then return } (n_a,\mu_a,M_{2,a},..., M_{k,a})$
\STATE \ \ $n=n_a+n_b$
\STATE \ \ $\delta=\mu_b-\mu_a$
\STATE \ \ $\mu=\mu_a+n_b\frac{\delta}{n}$
\STATE \ \ $\textbf{for } p=2 \to k$
\STATE \ \ \ \ $M_p=M_{p,a}+M_{p,b}+\sum\limits_{j=1}^{p-2}{p \choose j}[(-\frac{n_2}{n})^j M_{p-j,a}+(\frac{n_1}{n})^j M_{p-j,b}]\delta + (\frac{n_1n_2}{n}\delta)^p[\frac{1}{n_2^{p-1}}-(\frac{-1}{n_1})^{p-1}]$
\STATE \ \ \textbf{return} $(n,\mu,M_2,..., M_k)$ \}
}
\end{algorithmic}
\end{algorithm}


%$$n=n_1+n_2$$
%$$\delta=\mu_2-\mu_1$$
%$$\mu=\mu_1+n_2\frac{\delta}{n}$$
%$$M^p_g=M^p_{g_1}+M^p_{g_2}+\sum\limits_{k=1}^{p-2}{p \choose k}[(-\frac{n_2}{n})^k M^{p-k}_{g_1}+(\frac{n_1}{n})^k M^{p-k}_{g_2}]\delta + (\frac{n_1n_2}{n}\delta)^p[\frac{1}{n_2^{p-1}}-(\frac{-1}{n_1})^{p-1}]$$

\subsection{Stable Covariance}\label{sec:covariance}

Covariance ($\frac{\sum\limits_{i=1}^{n}(x_i-\bar{x})(y_i-\bar{y})}{n-1}$) is one of the basic bivariate statistics. It measures the strength of correlation between two data sets. A common textbook one-pass technique rewrites the equation as $\frac{1}{n-1}\sum\limits_{i=1}^{n}x_iy_i-\frac{1}{n(n-1)}\sum\limits_{i=1}^{n}x_i\sum\limits_{i=1}^{n}y_i$. Similar to the one-pass rewrite of variance described in Section~\ref{sec:highorder}, this rewrite also suffers from numerical errors and can produce grossly inaccurate results. 
%Unfortunately, such a rewrite is still being used in the distributed database platform~\cite{teradata} and suggested in~\cite{nips06} for MapReduce based covariance computation.

%Similar to variance, covariance suffers a pitfall of the textbook one-pass algorithm: people often use the rewrite $\frac{1}{n-1}\sum\limits_{i=1}^{n}x_iy_i-\frac{1}{n(n-1)}\sum\limits_{i=1}^{n}x_i\sum\limits_{i=1}^{n}y_i$ to compute covariance. This textbook one-pass algorithm is known to be unstable. Unfortunately, it is still being used in the distributed database platform~\cite{teradata} and suggested in~\cite{nips06} for MapReduce based covariance computation.

In SystemML, we adapt the technique proposed by Bennett {\em et al.}~\cite{cm} to the MapReduce setting. This algorithm computes the partial aggregates in the mappers and combines the partial aggregates in a single reducer. The update rule for computing $C=\sum\limits_{i=1}^{n}(x_i-\bar{x})(y_i-\bar{y})$ is shown below. Note that $\varoplus$ represents addition through \textsc{KahanIncrement}. The difference between the update rule used in SystemML and the original update rule in~\cite{cm} is that SystemML uses \textsc{KahanIncrement} for the updates instead of the basic addition. In Section~\ref{sec:comparison_exp}, we will empirically compare these two methods.
%The similar update rule is also used in HIVE for computing covariance, but again HIVE only applies basic addition instead of the more stable \textsc{KahanIncrement}. Later in Section~\ref{sec:comparison_exp}, we will compare SystemML against HIVE for covariance and Pearson correlation.

\begin{small}
\begin{equation}
\begin{split}
n=& n_a+n_b, \;\; \delta_x=\mu_{x,b}-\mu_{x,a}, \;\; \mu_x=\mu_{x,a}\varoplus n_b\frac{\delta_x}{n} \\
\delta_y=& \mu_{y,b}-\mu_{y,a}, \;\; \mu_y=\mu_{y,a}\varoplus n_b\frac{\delta_y}{n} \\
C =& C_a\varoplus C_b\varoplus \frac{n_an_b}{n}\delta_x\delta_y
\end{split}
\label{eq:coveq}
\end{equation}
\end{small}


