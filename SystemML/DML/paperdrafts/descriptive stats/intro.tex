
The growing need to analyze massive data sets has led to an increased interest in implementing machine learning algorithms on MapReduce~\cite{nips06,gillick2008mapreduce}. 
SystemML~\cite{systemml} is an Apache Hadoop based system for large scale machine learning, developed at IBM Research. In SystemML, machine learning algorithms are expressed as scripts written in a high-level language, called DML, with linear algebra and mathematical primitives. SystemML compiles these scripts, applies various optimizations based on data and system characteristics, and translates them into efficient runtime on MapReduce. %SystemML is part of the big data analytics solution offered in IBM Infosphere BigInsights~\cite{biginsights} (can I say that yet???). 
A wide variety of machine learning techniques can be expressed in SystemML, including classification, clustering, regression, matrix factorization, and ranking. 

Besides complex machine learning algorithms, SystemML also provides powerful constructs to compute \textit{descriptive statistics}. 
%Descriptive statistics are used to quantitively describe the basic features of data. 
%They are in direct comparison to inferential statistics that infer general conclusions from the sample data. 
%They are the foundation of virtually every quantitative analysis of data. 
Descriptive statistics primarily include {\em univariate analysis} that deals with a single variable at a time, and {\em bivariate analysis} that examines the degree of association between two variables. Table~\ref{tab:stats} lists the descriptive statistics that are currently supported in SystemML. In this paper, we describe our experience in addressing the two major challenges when implementing descriptive statistics in SystemML -- (1) numerical stability while operating on large data sets in the distributed setting of MapReduce; (2) efficient implementation of order statistics in MapReduce.

%\begin{table}[t]
%\centering
%\begin{tabular}{|c|l|}
%\hline
%&Sum, Mean, Harmonic mean, Geometric mean, Mode,\\
% &  Min, Max, Range, Median, Quantiles, Inter-quartile mean, \\ 
%Univariate & Variance, Standard deviation, Coefficient of variation, \\
%& Central moment, Skewness, Kurtosis, Standard error of mean,\\
%& Standard error of skewness, Standard error of kurtosis\\
%\hline
%Bivariate & Covariance, Pearson's R, Chi-squared coefficient, Cramer's V\\
%&  Eta, ANOVA F-measure, Spearman correlation\\
%\hline
%\end{tabular}
%\Crunch
%\caption{Descriptive statistics supported in SystemML}
%\BigCrunch
%\label{tab:stats}
%\BigCrunch
%\end{table}

\begin{table}[t]
\centering
\caption{Descriptive statistics supported in SystemML}
%\BigCrunch
\label{tab:stats}
%\BigCrunch
\begin{tabular}{|m{0.5in}|m{2.6in}|}
%\renewcommand {\tabularxcolumn}[1]{{>\arraybackslash}m{#1}}
%\begin{tabularx}{\textwidth}{c|p{2.4in}}
\hline
\multirow{2}{*}{{\bf Univariate}} & \vspace{0.02in} {\bf Scale variable:} Sum, Mean, Harmonic mean, Geometric mean, 
                                       Min, Max, Range, Median, Quantiles, Inter-quartile mean,
                                       Variance, Standard deviation, Coefficient of variation,
                                       Central moment, Skewness, Kurtosis, Standard error of mean,
                                       Standard error of skewness, Standard error of kurtosis \vspace{0.02in} \\ %\cline{2-2}
                                  & \vspace{0.02in} {\bf Categorical variable:} Mode, Per-category frequencies \vspace{0.02in}\\ 
\hline
\hline
\multirow{3}{*}{{\bf Bivariate}} & \vspace{0.02in} {\bf Scale-Scale variables:} Covariance, Pearson correlation  \\ %\cline{2-2}
                                 & \vspace{0.02in} {\bf Scale-Categorical variables:} Eta, ANOVA F measure \vspace{0.02in} \\ %\cline{2-2}
                                 & \vspace{0.02in} {\bf Categorical-Categorical variables:} Chi-squared coefficient, Cramer's V, Spearman correlation \\
\hline
\end{tabular}
\BigCrunch
\BigCrunch
\end{table}

%Most of descriptive statistics, except for order statistics, can be expressed in a certain summation form, therefore seeming easy to be implemented in MapReduce. However, straightforward implementations can often lead to disasters in numerical accuracy, due to overflow, underflow and round-off errors associated with finite precision arithmetic. The numerical stability problem gets more serious with the increasing need to analyze larger volumes of data. However, numerical stability issue is largely ignored in the context of large scale data processing. The popular Hadoop-based systems PIG~\cite{pig}, HIVE~\cite{hive} and JAQL~\cite{jaql} are still using dangerous naive implementations for sum, mean and variance. Surprisingly, numerically unstable algorithms are even in use in well-known distributed database products~\cite{netezza, teradata}. In SystemML, numerical stability is a major priority. In this paper, we first share our experience in the endeavor of ensuring numerical stability of the descriptive statistics, and call for attention to this important issue for large scale data processing.

Most of descriptive statistics, except for order statistics, can be expressed in certain summation form, and hence it may seem as if they are trivial to implement on MapReduce. However, straightforward implementations often lead to disasters in numerical accuracy, due to overflow, underflow and round-off errors associated with finite precision arithmetic. Such errors often get magnified with the increasing volumes of data that is being processed. However in practice, the issue of numerical stability is largely ignored, especially in the context of large scale data processing. Several well-known and commonly used software products still use numerically unstable implementations to compute several basic statistics. For example, McCullough and Heiser highlighted in a series of articles that Microsoft Excel suffers from numerical inaccuracies for various statistical procedures~\cite{mccullough2008accuracy,mccullough1999accuracy,mccullough2002accuracy,mccullough2005accuracy}. In their studies, they conducted various tests related to univariate statistics, regression, and Monte Carlo simulations using the NIST Statistical Reference Datasets (StRD)~\cite{nist}. They note that numerical improvements were made to univariate statistics only in the recent versions of Excel~\cite{mccullough2008accuracy}. In the context of large-scale data processing, Hadoop-based systems such as PIG~\cite{pig} and HIVE~\cite{hive} still use numerically unstable implementations to compute several statistics. PIG as of version 0.9.1 supports only the very basic \textit{sum} and \textit{mean} functions, and neither of them use numerically stable algorithms. The recent version of HIVE (0.7.1) supports more statistics including \textit{sum}, \textit{mean}, \textit{variance}, \textit{standard deviation}, \textit{covariance}, and \textit{Pearson correlation}. While \textit{variance}, \textit{standard deviation}, \textit{covariance}, and \textit{Pearson correlation} are computed using stable methods, both \textit{sum} and \textit{mean} are computed using numerically unstable methods. These examples highlight the fact that the issue of numeric stability has largely been ignored in practice, in spite of its utmost importance.

%{\bf a mention of numerical consistency in parallel env?}

%algorithms are even in use in well-known distributed database products~\cite{netezza, teradata}. 

In this paper, we share our experience in achieving numerical stability as well as scalability for descriptive statistics on MapReduce, and bring the community's attention to the important issue of numerical stability for large scale data processing. Through a detailed set of experiments, we demonstrate the scalability and numerical accuracy achieved by SystemML. We finally conclude by highlighting the lessons we have learned in this exercise. 

%In SystemML, numerical stability is a major priority. In this paper, we first share our experience in the endeavor of ensuring numerical stability of the descriptive statistics, and call for attention to this important issue for large scale data processing.

%Different from the other descriptive statistics, order statistics, used in median and quantiles, impose an order on all the input data, thus do not fit in the natural computation model of MapReduce. We will demonstrate how the existing distributed order statistics algorithms do not fit in the MapReduce environment and describe how we implement efficient order statistics in SystemML. 

%\begin{table*}[t]
%\centering
%\begin{tabular}{|c|c|l|}
%\hline
%& Distribution & Min, Max, Range, Quantiles \\
%\cline{2-3}
%Univariate & Central Tendency & Mean, Harmonic mean, Geometric mean, Standard error of mean, Median, Inter-quartile mean, Mode\\
%\cline{2-3} 
%Analysis & Dispersion & Variance, Standard deviation, Coefficient of variation \\
%\cline{2-3}
%& Shape & Central moment, Skewness, Standard error of skewness, Kurtosis, Standard error of kurtosis\\
%\hline
%& Scale by Scale & Pearson�s R\\
%\cline{2-3}
%Bivariate & Categorical by Categorical & Chi-squared coefficient, Cramer�s V \\
%\cline{2-3}
%Analysis & Scale by Categorical & Eta, ANOVA F-measure \\
%\cline{2-3}
%& Ordinal by Ordinal & Spearman correlation \\
%\hline
%\end{tabular}
%\caption{Descriptive statistics supported in SystemML}
%\label{tab:stats}
%\end{table*}
