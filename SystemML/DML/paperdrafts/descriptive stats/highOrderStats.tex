\subsection{Stable Higher-Order Statistics}
\label{sec:highorder}

We now describe our stable algorithms to compute higher-order statistics, such as variance, skewness and kurtosis. The core computation is to calculate the $p^{th}$ central moment $m_p=\frac{1}{n}\sum\limits_{i=1}^{n}(x_i-\bar{x})^p$. Central moment can be used to describe higher-order statistics. For instance, variance $\sigma^2=\frac{n}{n-1}m_2$, skewness $g_1=\frac{m_3}{m_2^{1.5}}\cdot(\frac{n-1}{n})^{1.5}$, and kurtosis $g_2=\frac{m_4}{m_2^{2}}\cdot(\frac{n-1}{n})^{2}-3$. The standard two-pass algorithm produces relatively stable results (for $m_2$, the relative error bound is $nu+n^2u^2\condnumber ^2$, where $\condnumber = (\frac{\sum\limits_{i=1}^{n}x_i^2}{\sum\limits_{i=1}^{n}(x_i-\bar{x})^2})^{\frac{1}{2}}$ is the condition number), but it requires two scans of data -- one scan to compute $\bar{x}$ and the second to compute $m_p$. A common technique (pitfall) is to apply a simple textbook rewrite to get a one-pass algorithm. For instance, $m_2$ can be rewritten as $\frac{1}{n}\sum\limits_{i=1}^{n}x_i^2-\frac{1}{n^2}(\sum\limits_{i=1}^{n}x_i)^2$. The sum of squares and sum can be computed in a single pass. However, this algebraical rewrite is known to suffer from serious stability issues resulting from cancellation errors when performing subtraction of two large and nearly equal numbers (relative error bound is $nu\condnumber ^2$). This rewrite, as we show later in Section~\ref{sec:univariate_exp}, may actually produce a negative result for $m_2$, thereby resulting in grossly erroneous values for variance and other higher-order statistics. 
%Surprisingly, this notoriously unstable algorithm is still used in a popular distributed database platform~\cite{teradata} -- emphasizing the need for immediate attention to numerical stability issues in large scale data processing.

%A pitfall that people often run into is to apply a simple textbook rewrite rule to get a one-pass algorithm. For example, $2^{nd}$ central moment is rewritten as $m_2=\sum\limits_{i=1}^{n}x_i^2-\frac{1}{n}(\sum\limits_{i=1}^{n}x_i)^2$, where the sum as well as sum of squares are computed in a single pass. Even though the rewrite is algebraically equivalent, it is known to suffer from serious stability issues resulting from cancellation errors while performing the substraction of two large nearly equal numbers.

%Besides simple summation, descriptive statistics contain a number of high order statistics, such as central moment, variance, skewness and kurtosis. The $k^{th}$ central moment is defined as $m_k=\frac{1}{n}\sum\limits_{i=1}^{n}(x_i-\bar{x})^k$. Central moment is also the building block for the other high order statistics. For example, variance can be written as $s^2=\frac{n}{n-1}m_2$ and skewness can be expressed as $g_1=\frac{m_3}{m_2^{1.5}}$. Therefore, to produce efficient and numerically stable high order statistics, we need an efficient and numerically stable central moment algorithm. The standard two-pass algorithm using stable summation produces the numerically stable central moment, but requires two scans of data: the first to produce the mean and the second to calculate the central moment. A pitfall people often run into is to apply a simple textbook rewrite on the definition of central moment to get a one-pass algorithm. For example, $2^{nd}$ central moment can be rewritten as $m_2=\sum\limits_{i=1}^{n}x_i^2-\frac{1}{n}(\sum\limits_{i=1}^{n}x_i)^2$. Then this textbook one-pass algorithm can calculate the sum and sum of squares in one scan of the data followed by simply arithmetic on the results. As elaborated in~\cite{numStabBook}, this seeming mathematically equivalent and more efficient algorithm can lead to numerical disasters, due to the severe cancellation problem when performing a substraction on two large and nearly equal numbers. The computed $2^{nd}$ central moment (or variance) is often zero and sometimes even negative. But surprisingly, this notoriously unstable algorithm is still in use in some database products~\cite{netezza, teradata}.

In SystemML, we use a stable MapReduce algorithm that is based on an existing technique~\cite{cm}, to incrementally compute central moments of arbitrary order. It makes use of an update rule (Equation~\ref{eq:cmeq}) that combines partial results obtained from two disjoint subsets of data (denoted as subscripts $a$ and $b$). Here, $n$, $\mu$, $M_p$ refer to the cardinality, mean, and $n\times m_p$ of a subset, respectively. Again, $\varoplus$ represents addition through \textsc{KahanIncrement}. When $p=2$, the algorithm has a relative error bound of $nu\condnumber$. While there is no formal proof to bound the relative error when $p > 2$, we empirically observed that this method produces reasonably stable results for skewness and kurtosis. In the MapReduce setting, the same update rule is used in each mapper to maintain running values for these variables, and to combine partial results in the reducer. Note that the update rule in~\cite{cm} uses basic addition instead of the more stable \textsc{KahanIncrement}. In Section~\ref{sec:comparison_exp}, we will evaluate the effect of \textsc{KahanIncrement} on the accuracy of higher-order statistics.
 
%Note that HIVE utilizes a similar update rule to compute variance and standard deviation. However, HIVE uses the basic addition for the updates instead of the more stable \textsc{KahanIncrement}. In Section~\ref{sec:comparison_exp}, we will evaluate the effect of \textsc{KahanIncrement} by comparing the algorithms used in SystemML and HIVE for variance and standard deviation.

%\begin{equation}
%M_p=M_{p,a}+M_{p,b}+\sum\limits_{j=1}^{p-2}{p \choose j}[(-\frac{n_2}{n})^j M_{p-j,a}+(\frac{n_1}{n})^j M_{p-j,b}]\delta + (\frac{n_1n_2}{n}\delta)^p[\frac{1}{n_2^{p-1}}-(\frac{-1}{n_1})^{p-1}]$
%\STATE \ \ \textbf{return} $(n,\mu,M_2,..., M_k)
%\label{eq:cmeq}
%\end{equation}
\Crunch
\begin{small}
\begin{equation}
\begin{split}
n=& n_a+n_b, \;\; \delta=\mu_b-\mu_a, \;\; \mu=\mu_a\varoplus n_b\frac{\delta}{n} \\
M_p =& M_{p,a}\varoplus M_{p,b}\varoplus \{\sum\limits_{j=1}^{p-2}{p \choose j} [(-\frac{n_b}{n})^j M_{p-j,a} \\
  & + (\frac{n_a}{n})^j M_{p-j,b}]\delta^j + (\frac{n_an_b}{n}\delta)^p[\frac{1}{n_b^{p-1}}-(\frac{-1}{n_a})^{p-1}]\}
\end{split}
\label{eq:cmeq}
\end{equation}
\end{small}
\Crunch

%We adopt the algorithm introduced in~\cite{cm} to numerically stably calculate arbitrary order central moment. As shown in Algorithm~\ref{algo:cm}, the core of the algorithm is to calculate the value of $M_k=\sum\limits_{i=1}^{n}(x_i-\bar{x})^k$. Similar to the Kahan Sum algorithm, this algorithm can also be easily adapted to the MapReduce environment, by letting mappers compute the partial values and a single reducer produce the final result.

%\begin{algorithm}
\caption{Central Moment} 
\label{algo:cm}
\begin{algorithmic}
\small{
\STATE \textbf{CentralMoment}(X: input data items, k: order)\{
\STATE \ \ $n=0$, $\mu=0$, $M_2=0$..., $M_k=0$
\STATE \ \ \textbf{for} $i = 1 \to X.size$ 
\STATE \ \ \ \ $(n, \mu, M_2,...,M_k)=\textbf{incM}(n, \mu, M_2,...,M_k,1, x_i, 0,...,0)$
%\STATE \ \ \ \ \ \ \ \ \ \ \ \ \ \ \ \ \ \ \ \ \ \ \ \ \ \ \ \ \ \ \ \ \ \ \ \ \ \ \ \ \ \ \  $1, x_i, 0,...,0)$
\STATE \ \ \textbf{return} $\frac{M^k}{n}$ \}
\STATE 
\STATE \textbf{incM}($n_a,\mu_a,M_{2,a},..., M_{k,a}$, // $1^{st}$ input
\STATE \ \ \ \ \ \ $n_b,\mu_b,M_{2,b},..., M_{k,b}$)\{ // $2^{nd}$ input
\STATE \ \ $\textbf{if }n_1=0\textbf{ then return } (n_b,\mu_b,M_{2,b},..., M_{k,b})$
\STATE \ \ $\textbf{if }n_2=0\textbf{ then return } (n_a,\mu_a,M_{2,a},..., M_{k,a})$
\STATE \ \ $n=n_a+n_b$
\STATE \ \ $\delta=\mu_b-\mu_a$
\STATE \ \ $\mu=\mu_a+n_b\frac{\delta}{n}$
\STATE \ \ $\textbf{for } p=2 \to k$
\STATE \ \ \ \ $M_p=M_{p,a}+M_{p,b}+\sum\limits_{j=1}^{p-2}{p \choose j}[(-\frac{n_2}{n})^j M_{p-j,a}+(\frac{n_1}{n})^j M_{p-j,b}]\delta + (\frac{n_1n_2}{n}\delta)^p[\frac{1}{n_2^{p-1}}-(\frac{-1}{n_1})^{p-1}]$
\STATE \ \ \textbf{return} $(n,\mu,M_2,..., M_k)$ \}
}
\end{algorithmic}
\end{algorithm}


%$$n=n_1+n_2$$
%$$\delta=\mu_2-\mu_1$$
%$$\mu=\mu_1+n_2\frac{\delta}{n}$$
%$$M^p_g=M^p_{g_1}+M^p_{g_2}+\sum\limits_{k=1}^{p-2}{p \choose k}[(-\frac{n_2}{n})^k M^{p-k}_{g_1}+(\frac{n_1}{n})^k M^{p-k}_{g_2}]\delta + (\frac{n_1n_2}{n}\delta)^p[\frac{1}{n_2^{p-1}}-(\frac{-1}{n_1})^{p-1}]$$

\subsection{Stable Covariance}\label{sec:covariance}

Covariance ($\frac{\sum\limits_{i=1}^{n}(x_i-\bar{x})(y_i-\bar{y})}{n-1}$) is one of the basic bivariate statistics. It measures the strength of correlation between two data sets. A common textbook one-pass technique rewrites the equation as $\frac{1}{n-1}\sum\limits_{i=1}^{n}x_iy_i-\frac{1}{n(n-1)}\sum\limits_{i=1}^{n}x_i\sum\limits_{i=1}^{n}y_i$. Similar to the one-pass rewrite of variance described in Section~\ref{sec:highorder}, this rewrite also suffers from numerical errors and can produce grossly inaccurate results. 
%Unfortunately, such a rewrite is still being used in the distributed database platform~\cite{teradata} and suggested in~\cite{nips06} for MapReduce based covariance computation.

%Similar to variance, covariance suffers a pitfall of the textbook one-pass algorithm: people often use the rewrite $\frac{1}{n-1}\sum\limits_{i=1}^{n}x_iy_i-\frac{1}{n(n-1)}\sum\limits_{i=1}^{n}x_i\sum\limits_{i=1}^{n}y_i$ to compute covariance. This textbook one-pass algorithm is known to be unstable. Unfortunately, it is still being used in the distributed database platform~\cite{teradata} and suggested in~\cite{nips06} for MapReduce based covariance computation.

In SystemML, we adapt the technique proposed by Bennett {\em et al.}~\cite{cm} to the MapReduce setting. This algorithm computes the partial aggregates in the mappers and combines the partial aggregates in a single reducer. The update rule for computing $C=\sum\limits_{i=1}^{n}(x_i-\bar{x})(y_i-\bar{y})$ is shown below. Note that $\varoplus$ represents addition through \textsc{KahanIncrement}. The difference between the update rule used in SystemML and the original update rule in~\cite{cm} is that SystemML uses \textsc{KahanIncrement} for the updates instead of the basic addition. In Section~\ref{sec:comparison_exp}, we will empirically compare these two methods.
%The similar update rule is also used in HIVE for computing covariance, but again HIVE only applies basic addition instead of the more stable \textsc{KahanIncrement}. Later in Section~\ref{sec:comparison_exp}, we will compare SystemML against HIVE for covariance and Pearson correlation.

\begin{small}
\begin{equation}
\begin{split}
n=& n_a+n_b, \;\; \delta_x=\mu_{x,b}-\mu_{x,a}, \;\; \mu_x=\mu_{x,a}\varoplus n_b\frac{\delta_x}{n} \\
\delta_y=& \mu_{y,b}-\mu_{y,a}, \;\; \mu_y=\mu_{y,a}\varoplus n_b\frac{\delta_y}{n} \\
C =& C_a\varoplus C_b\varoplus \frac{n_an_b}{n}\delta_x\delta_y
\end{split}
\label{eq:coveq}
\end{equation}
\end{small}

