\begin{table*}[t]
\centering
\caption{Example hops in SystemML: $x_{ij}$, $y_{ij}$ are cells in matrices $X$, $Y$, respectively.}
\label{tab:hop}
\begin{small}
\footnotesize
\begin{tabular}{|l|l|m{3.2in}|l|}
\hline
{\bf HOP Type} & {\bf Notation} & {\bf Semantics} & {\bf Example in Table~\ref{tab:notation}} \\
\hline
{\em Binary} &$b(op): X, Y$  & for each $x_{ij}$ and $y_{ij}$, perform $op(x_{ij}, y_{ij})$, where $op$ is $*, +, -, /$ etc. & $b(*): X, Y$ \\ \hline
{\em Unary} &$u(op): X$ & for each $x_{ij}$, perform $op(x_{ij})$, where $op$ is $log$, $sin$ etc. & $u(log): X$\\ \hline
{\em AggregateUnary}& $au(aggop, dimension): X$ & apply $aggop$ for the cells in dimension, where $aggop$ is $\sum$, $\prod$ etc, and $dimension$ is  
$row$ (row wise), $col$ (column wise) or $all$ (the whole matrix). & $au(\sum,row): X$\\ \hline
{\em AggregateBinary}& $ab(aggop, op): X, Y$ & for each $i,j$, perform $aggop(\{op(x_{ik}, y_{kj})| \forall k\})$, where $op$ is $*, +, -, /$ etc, and $aggop$ is 
$\sum$, $\prod$ etc. & $ab(\sum, *): X, Y$ \\  \hline
{\em Reorg} & $r(op): X$ & reorganize elements in a matrix, such as transpose ($op=T$). & $r(T): X$ \\ \hline
{\em Data} & $data(op): X$ & read ($op=r$) or write ($op=w$) a matrix. & \\ 
\hline
\end{tabular}
\end{small}
\SmallCrunch
\end{table*}

The HOP component takes the parsed representation of a statement block as input, and produces a HOP-Dag representing the data flow. 
%We first describe the hops available in \systemmltext\ along with
%their semantics, and then describe the construction of HOP-Dag.


\customizedfigInCol
{figures/hoplop.eps} 
{HOP-Dag, LOP-Dag and Runtime of the while Loop Body in Figure~\ref{fig:programanalysis}}
{fig:hoplop}
{5in}

\noindent{\bf Description of hops:} 
Each hop in the HOP-Dag has one or more input(s), performs an operation or transformation, 
and produces output that is consumed by one or more subsequent hops. Table~\ref{tab:hop} 
lists some example hops supported in \systemmltext\ along with their 
semantics\footnote{Table~\ref{tab:hop} describes the semantics of hops in terms of 
matrices. Semantics of hops for scalars are similar in spirit.}. In addition, the instantiation of hops from the \dmlr\ parsed representation is exemplified in Table~\ref{tab:notation}. 
Consider the matrix multiplication \texttt{Z=X\mmult Y} as an instance, an {\em 
AggregateBinary} hop is instantiated with the binary operation $*$ and the aggregate operation $\sum$. The semantics of this hop instance, denoted by $ab(\sum,*)$, is to compute, $\forall i,j, \sum_k ({x_{i,k} * y_{k,j}})$. 


%Consider the cell-wise multiplication \texttt{Z=X*Y} as an instance, a {\em Binary} hop is instantiated with binary operation $*$, and $X$ and $Y$ as operands. The semantics of this binary hop is to compute $x_{i,j} * y_{i,j}$ for every pair of corresponding cells in the two matrices. For the more complex matrix multiplication \texttt{Z=X\mmult Y}, an {\em AggregateBinary} hop is instantiated with the binary operation $*$ and the aggregate operation $\sum$. The semantics of hop instance, denoted by $ab(\sum,*)$, is to compute, $\forall i,j, \sum_k ({x_{i,k} * y_{k,j}})$. 

%To understand the instantiation of hops from the \dmlr\ parsed representation,
%let us consider the cell-wise division of matrices $C$ and $D$ shown in
%Figure~\ref{fig:example}. a {\em Binary} hop is instantiated with
%binary operation $/$, and $C$ and $D$ as operands. The semantics of this
%binary hop is such that it computes $c_{i,j} / d_{i,j}$ for every
%pair of corresponding cells in the two matrices. For a more complex hop 
%such as {\em AggregateBinary}, let us consider the  matrix multiplication 
%$W \mmult H$ in Script~\ref{scpt:gnmf}. For this purpose an
%{\em AggBinary} hop is instantiated
%with the binary operation $*$ and the aggregate operation $+$. 
%
%Data hops can be transient or persistent with respect to the lifetime of a script.


\noindent{\bf Construction of HOP-Dag:} 
The computation in each statement block is represented as one HOP-Dag~\footnote{Statement blocks for control structures such as {\it while} loops have additional HOP-Dags, e.g. for representing predicates.}. Figure~\ref{fig:hoplop}(a) shows the HOP-Dag for the body of the \textit{while} loop statement block in Figure~\ref{fig:programanalysis} constructed using the hops in Table~\ref{tab:hop}. Note how multiple statements in a statement block have been combined into a single HOP-Dag. The HOP-Dag need not be a connected graph, as shown in Figure~\ref{fig:hoplop}(a).

The computation \texttt{t(W)\mmult W} in the statement block is represented 
using four hops -- a {\it data(r):$W$} hop that reads $W$ is fed into a Reorg hop 
{\it r($T$)} to perform the matrix transposition, which is then fed, along with the {\it 
data(r):$W$} hop, into an AggregateBinary hop {\it $ab(\sum,*)$} to perform the matrix 
multiplication. 
%Note that a HOP-Dag does not have to be connected as shown in 
%Figure~\ref{fig:hoplop}(a).

%The hop {\it data(r):$W$} feeds into the Reorg hop {\it r($T$)}, which in turn feeds into 
%the AggregateBinary hop {\it $ab(\sum,*)$} together with the {\it data(r):$W$} hop to 
%represent the \texttt{t(W)\mmult W} part of the statement block. Note that a HOP-Dag does 
%not have to be connected as shown in Figure~\ref{fig:hoplop}(a).

The grayed \textit{data(r)} hops represent the live-in variables for matrices
$W$, $H$, and $V$, and the scalar $i$ at the beginning of an
iteration\footnote{The max\_iteration variable is used in the HOP-Dag
for the \concept{while} loop predicate.}. The grayed \textit{data(w)} hops
represent the live-out variables at the end of an iteration that need
to be passed onto the next iteration. These data hops -- which are
transient -- implicitly connect HOP-Dags of different statement blocks
by mapping the transient \textit{data(w)} hops (sinks) of
one statement block to the transient \textit{data(r)} hops (sources) of the next statement 
block, or the next iteration of the \concept{while} loop.

%\begin{figure}[t]
%\center
%{\includegraphics[width=2.6in]{figures/hopsdag.eps}}
%\caption{HOPDag for While Loop Body in Figure~\ref{fig:programanalysis}.}
%\label{fig:hopdag}
%\end{figure}
